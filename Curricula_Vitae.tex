%%%%%%%%%%%%%%%%%%%%%%%%%%%%%%%%%%%%%%%%%
% Sebastian Hultstrands CV 
% Sebastian@storm-net.se
%
% This template has been downloaded from:
% http://www.LaTeXTemplates.com
%
% Original author:
% Adrien Friggeri (adrien@friggeri.net)
% https://github.com/afriggeri/CV
%
% License:
% CC BY-NC-SA 3.0 (http://creativecommons.org/licenses/by-nc-sa/3.0/)
%
% Important notes:
% This template needs to be compiled with LuaLaTeX
%
%%%%%%%%%%%%%%%%%%%%%%%%%%%%%%%%%%%%%%%%%

\documentclass[]{friggeri-cv} % Add print as an option into the square bracket to remove colors from this template for printing
\usepackage{pdfpages}

%\addbibresource{bibliography.bib} % Specify the bibliography file to include publications

\begin{document}

\header{Sebastian}{Hultstrand}{Webbprogrammerare / Mjukvaruutvecklare} % Your name and current job title/field

%----------------------------------------------------------------------------------------
%	SIDEBAR SECTION
%----------------------------------------------------------------------------------------

\begin{aside} % In the aside, each new line forces a line break
\section{kontakt}
Sandrevelsvägen 7
422 50, Hisings backa
Sverige
~
0768553735
~
\href{mailto:sebastian@storm-net.se}{sebastian@storm-net.se}
\href{https://se.linkedin.com/pub/sebastian-hultstrand/60/51a/151}{linkedin}
\href{https://github.com/cruzzan}{github: Cruzzan}
\section{språk}
svenska och engelska
\section{programmering}
Java, Ruby, C++, PHP, LaTeX, JavaScript, CSS3 \& HTML5
\section{verktyg utveckling}
JetBrains suit, Android Studio, Gradle
\section{verktyg testning}
PhpUnit, Gherkin, JUnit
\section{verktyg management}
Jenkins, Bamboo, Stash, git, Jira, Confluence, Redmine
\section{verktyg hosting}
Apache2, Tomcat, nginx
\end{aside}

%----------------------------------------------------------------------------------------
%	EDUCATION SECTION
%----------------------------------------------------------------------------------------

\section{utbildning}

\begin{entrylist}
%------------------------------------------------
\entry
{2012--2015}
%{Fil. Kand {\normalfont mjukvaruingenjör}} % Lägg till efter examen är tagen
{Kandidatprogram Webbprogrammering 180hp.}
{Blekinge Tekniska Högskola, Karlskrona}
%{Område: webbprogrammering} % Lägg till efter examen är tagen
{}
%------------------------------------------------
\end{entrylist}

%----------------------------------------------------------------------------------------
%	WORK EXPERIENCE SECTION
%----------------------------------------------------------------------------------------

\section{erfarenhet}

\begin{entrylist}
%------------------------------------------------
\entry
{2015--Now}
{Webdoc, Advance/TeleComputing}
{Göteborg}
{\emph{Utvecklare, system arkitekt} \\
Utöver vanligt underhåll jobbade jag på ett antal större projekt till produkten.
\begin{itemize}
\item Ny utskriftslösning för att ersätta beroende på Appletes, genom en integration mot skrivarprogramvara för websystem (\href{http://qz.io}{QZ tray})
\item Ny fil-/dokumenthanteringsmodul.
\item Storskalig databas ombyggnation.
\item CE certifiering av systemet.
\end{itemize}
Bland mina övriga ansvarsområden fanns också; Anvsar för demonstrationer av förändringar inför release, samt hantering av merge och handpåläggning inför regression och release. \\
Under min tid har jag också varit en dela av utvecklingen av processer samt arbetsmetoder för vårt semi-CI arbetsflöde.\\

}\\
%------------------------------------------------
\entry
{2011--2014}
{Atlan solutions}
{Kungsbacka, Halland}
{\emph{Webbprogrammerare, systemarkitekt, supporttekniker} \\
Jag arbetade primärt på två projekt, ett journalsystem; \href{http://atlan.se/}{WebDoc} och ett arbetshanterings system; Roofnet, för företaget \href{http://www.roofia.se/}{Roofia}. Jag arbetade som assisterande utvecklare samt support tekniker på WebDoc och som huvudsaklig utvecklare, systemarkitekt samt supporttekniker för Roofnet. Jag arbetade också mot ett antal mindre kunder som vi agerade webbhotell för. \\
\begin{itemize}
	\item 2011--2012 Heltid
	\item 2012--2014 Deltid, utveckling av Roofnet vid sidan av studierna.
	\item 2013 Sommarjobb
	\item 2014 Sommarjobb
\end{itemize}
}
%------------------------------------------------
\entry
{2013--2015}
{SIS, Blekinge studentkår}
{Karlskrona, Blekinge}
{\emph{Ideellt arbete} \\
Under perioden 2013--2015 satt jag som basgruppsansvarig för WebDev inom SIS (Sektionen för Internetbaserad Socialisering) på Blekinge studentkår. Där arbetade jag primärt med vidareutveckling och underhåll av befintliga system, däribland medlemssystemet. Jag hjälpte även till med att utveckla en väldigt primitiv PR hemsida för Blekinge tekniska högskolas introduktionsvecka; \href{http://www.nollningen.nu}{Nollningen.nu}. }
%------------------------------------------------
\end{entrylist}
\newpage
%----------------------------------------------------------------------------------------
%	AWARDS SECTION
%----------------------------------------------------------------------------------------

\section{utmärkelser}

\begin{entrylist}
%------------------------------------------------
\entry
{2010}
{1\textsuperscript{a} pris för bästa tjänst}
{Ung företagsamhet, Älvsborg}
{Priset tilldelades vårt uf företag; Storm Net, som jag agerade VD för. Vi tillhandahöll tjänster inom IT-support och LAN-party anordnande.}
%------------------------------------------------
\entry
{2010}
{2\textsuperscript{a} pris för bästa webbplats}
{Ung företagsamhet, Älvsborg}
{Hemsidan för vårt UF företag som jag byggde tillsammans med en vän som hjälpte mig med den grafiska designen.}
%------------------------------------------------
\end{entrylist}

%----------------------------------------------------------------------------------------
%	PORTFOLIO SECTION
%----------------------------------------------------------------------------------------

\section{portfolio}

\begin{entrylist}
%------------------------------------------------
\entry
{2014}
{Nollningen.nu}
{Blekinge studentkår}
{\href{http://www.nollningen.nu}{Nollningen.nu} byggdes tillsammans med en grupp studenter vid BTH för att agera som PR för introduktionsveckan.}
%------------------------------------------------
\end{entrylist}

%----------------------------------------------------------------------------------------
%	ABOUT ME SECTION
%----------------------------------------------------------------------------------------

\section{mer om mig}
\textbf{född:} 1992 \textbf{utrikes:} Bodde i Australien 2005--2008 \textbf{gymnasium:} Kunskapskällan, Herrljunga \textbf{ideellt arbete:} Satt som sekreterare i BITS utbildningsförening 2013--2014 samt vice-ordförande 2014--2015. Arbetade i studentpuben för Blekinge studentkår under utbildningstiden. Var delaktig i skapandet samt satt som ledamot i Studentkultur i Tiden 2013--2015. Satt som ledamot i Utbildningsrådet samt utbildningsutskottet 2014--2015.

%----------------------------------------------------------------------------------------
%	INTERESTS SECTION
%----------------------------------------------------------------------------------------

\section{intressen}

\textbf{professionellt:} nya tekniker, Linux miljöer, databashantering, webbsystem, webbintegrering, CI metodik, säkerhet, arkitekturell design. \textbf{fritid:} golf, matlagning, minecraft, paintball, skidåkning, koda koden

%----------------------------------------------------------------------------------------
%	REFERENCES SECTION
%----------------------------------------------------------------------------------------

\section{referenser}
\begin{entrylist}
	\entry
	{}
	{Niclas Hugosson}
	{Produktchef TeleComputing, grundare av WebDoc}
	{Niclas var min chef på Atlan och vi arbetade väldigt nära varandra under 2011--2012. På Advance/TeleComputing agerade han produktchef, med ansvar för utveckling och framtagande av produkter. \textbf{Kontakt:} \href{mailto:niclas.hugosson@telecomputing.se}{niclas.hugosson@telecomputing.se}, 0732-663399}
	\entry
    {}
    {Karl Annerhult Lagrell}
    {Utvecklings-team ledare på WebDoc, Advance/TeleComputing}
    {Karl är team ledare för utvecklarna på WebDoc och han är alltså en av dom som jag jobbade ihop med på WebDoc på senare år. \textbf{Contact:} \href{mailto:karl.annerhult@advance.se}{karl.annerhult@advance.se}}
	\entry
	{}
	{Martin Bagge}
	{Plattformsutvecklare på Procera Networks}
	{Martin arbetade ideellt samtidigt som jag för SIS och vi byggde en del saker tillsammans. \textbf{Kontakt:} \href{mailto:martin@bagge.nu}{martin@bagge.nu}, 0706-238104 }
\end{entrylist}

%----------------------------------------------------------------------------------------
%	PUBLICATIONS SECTION
%----------------------------------------------------------------------------------------

%\section{publications}

%----------------------------------------------------------------------------------------
%	APPENDIX SECTION
%----------------------------------------------------------------------------------------

\section{bilagor}
\begin{enumerate}
	\item \textbf{Cambridge CAE certifikat} utfärdat för kunskap av engelska som andra språk.
	\item \textbf{Intyg ansvarig alkoholservering} utfärdat efter genomgörande av utbildning inom alkoholservering, detta var tvunget för att kunna jobba i baren på studentpuben.
\end{enumerate}
\newpage
\appendix
\includepdf[pages=-]{appendix/CAE.pdf}
\includepdf[pages=-]{appendix/AAS.pdf}

%----------------------------------------------------------------------------------------

\end{document}